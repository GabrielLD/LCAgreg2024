\begin{headerBlock}
\chapter{Oxydoréduction}
\label{LC_Oxydoreduction_SPCL}
 \end{headerBlock}

%%%%%%%%%%%%%%%%%%%%%%%%%%%%%%%%%%%%%%%%%%%%%%%%%%%%
%%%% Références


%%%%%%%%%%%%%%%%%%%%%%%%%%%%%%%%%%%%%%%%%%%%%%%%%%%%
%%%% Plan
\begin{reportBlock}{Bibliographie}

\begin{center}
\begin{tabularx}{\textwidth}{| X | X | c | c |}\hline
Titre & Auteur(s) & Editeur (année) & ISBN \\ \hline
 Chimie PCSI Chap 12 & J-B Baudin & Dunod (2019) &  \\ 
 \hline
 \url{https://spcl.ac-montpellier.fr/moodle/course/view.php?id=61&section=4} & Académie Montpellier & & \\
 \hline
 BO Terminale STL/SPCL & Sujet bac pour l'hydrolyse de l'eau dans l'ISS & & \\
\end{tabularx}
\end{center}

\end{reportBlock}

\begin{reportBlock}{Plan détaillé}

\underline{Niveau} : Terminale STL-SPCL \\

\section*{Introduction pédagogique}

\paragraph*{Prérequis}
\begin{itemize}
\item titrages
\item Réactions acide/base
\item potentiel électrique
\end{itemize}

\paragraph*{Contexte :}
Premier trimestre 


\paragraph*{Notions importantes}

\begin{itemize}
\item 
\end{itemize}

\paragraph*{Objectifs}

\begin{itemize}
\item 
\end{itemize}

\paragraph*{Difficultés}

\begin{itemize}
\item formule de Nernst : mathématique
\item 

\end{itemize}

\section*{Introduction }
\textcolor{blue}{Manipulation introductive :} Pile Daniell. On observe une différence de potentiel, pourquoi ?
\section{Oxydants et réducteurs (5min)}

\subsection{Définitions}
\textcolor{green}{Oxydant :} espèce capble de capturer un ou plusieurs électrons. Ex : Zn$^{2+}$ ou Cu$^{2+}$.\\
\textcolor{green}{Réducteur :} espèce capable de capter un ou plusieurs électrons. Ex : Cu(s) ou Zn(s).

\subsection{Couple oxydant-réducteur}
A tout oxydant est associé un réducteur qui sont reliés par une demi-équation électronique.\\
Exemple : Cu$^{2+}$ + 2e$^-$ = Cu(s).\\

Remarque : on peut écrire de manière plus générale pou un couple oxydant (Ox)/réducteur (Red) :
\begin{equation}
    Ox + n\times e^- = Red
\end{equation}
\\
Remarque : il faut équilibrer les réactions redox. Exemple du couple Mn$^{2+}$/MnO$_4^{2-}$.

\section{Réaction d'oxydoréduction (5min)}
\subsection{Définition}
Dans une réaction d'oxydoréduction, les réactifs sont un oxydant et un réducteur de deux couples différents : il y a un transfert d'électrons.\\

Exemple : \begin{itemize}
    \item Al$^{3+}$/Al(s)
    \item Zn$^{2+}$/Zn(s)
\end{itemize}
On écrit les deux demi-équations redox et on obtient les équations bilans :
\begin{equation}
    2Al^{3+}(aq) + 3Zn(s) = 2Al(s) + 3Zn^{2+}(aq)
\end{equation}

\subsection{Réactions en milieu acide/basique}
\subsubsection{En milieu acide}
Exemple de MnO$_4^-$/Mn$^{3+}$ avec O$_2$/H$_2$O$_2$. Cette réaction  nécessite des ions H$^+$ pour avoir lieu donc elle doit se passer en milieu acide.
\subsubsection{En milieu basique}
Exemple de MnO$_4^-$/Mn$^{3+}$ avec CNO$^-$/CN$^-$. Elle nécessite des ions HO$^-$ pour se faire donc un milieu basique.

\section{Potentiel d'un couple redox (5-10min)}
\subsection{Principe de la mesure : retour sur la pile}
\begin{itemize}
    \item Une pile est constitué de deux demi-piles reliées par un pont salin,
    \item Chacune d'entre elle est constituée d'un couple redox auquel est associé un potentiel E (en V),
    \item Ce potentiel s'obtient en mesurant la différence de potentiel entre une demi-pile que l'on souhaite étudier et une autre demi-pile au potentile fixé/connu : l'électrode standard à hydrogène (théorique, E=0V) ou l'Electrode au Calomel Saturé (pratique, E=0.24V)
\end{itemize}
\subsection{Potentiel standard d'un couple rédox}
Potentiel d'un couple dans les conditions standard telles que : P=1bar et T=25$\degree$C.
\subsection{Relation de Nernst}
Relation qui permet de calculer le potentiel d'un couple redox. Pour une équation redox $\alpha$Ox +ne$^-$ = $\beta$Red, la relation de Nerst s'écrit :
\begin{equation}
    E(Ox/red) = E^0(Ox/red) + \frac{RT}{nF}ln\left( \frac{[Ox]_{eq}^\alpha}{[Red]_{eq}^\beta}\right)
\end{equation}

\section{Titrage d'oxydo-réduction}
Titrage dont la réaction de support est une réaction d'oxydo-réduction.

\textcolor{blue}{Expérience quantitative :}Titrage des ions Fe$^{2+}$ par les ions Ce$^{4+}$. Ca sert pour le vin.

\section*{Conclusion} 

\end{reportBlock}

