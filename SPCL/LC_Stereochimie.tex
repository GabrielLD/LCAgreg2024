\begin{headerBlock}
\chapter{Stéréochimie de configuration}
\label{LC_Stéréochimie}
 \end{headerBlock}

%%%%%%%%%%%%%%%%%%%%%%%%%%%%%%%%%%%%%%%%%%%%%%%%%%%%
%%%% Références


%%%%%%%%%%%%%%%%%%%%%%%%%%%%%%%%%%%%%%%%%%%%%%%%%%%%
%%%% Plan
\begin{reportBlock}{Bibliographie}

\begin{center}
\begin{tabularx}{\textwidth}{| X | X | c | c |}\hline
Titre & Auteur(s) & Editeur (année) & ISBN \\ \hline
\url{http://ressources-stl.fr/wp-content/uploads/2020/08/Structure-spatiale-des-especes-chimiques.pdf} & Ressource STL & ~ & ~ \\
\hline
\url{https://truejulosdu13.github.io/assets/organic_chemistry_lessons/Cours2.pdf} & Jules Schleinitz &  ~ & ~ \\ \hline
 \url{https://spcl.ac-montpellier.fr/moodle/course/view.php?id=61} & Académie de Montpellier & Chapitre 9 & ~ \\ 
 \hline
 Compte-rendu leçon Thibaut LC03 & & & \\
 \hline
\end{tabularx}
\end{center}

\end{reportBlock}

\begin{reportBlock}{Plan détaillé}

\underline{Niveau} : Tle STL - SPCL \\

\section*{Introduction pédagogique}


\paragraph*{Prérequis}
\begin{itemize}
\item représentation de Lewis, Cram
\item formalismes des flèches courbes
\end{itemize}

\paragraph*{Contexte :}
Place de la leçon : dernier cours de chimie.s

\paragraph*{Notions importantes}

\begin{itemize}
\item Stéréoisomèrie (enantiomère, diastéréoisomère)
\item stabilité des carbocations,
\item polarimétrie de Laurent
\end{itemize}

\paragraph*{Objectifs}

\begin{itemize}
\item géométrie carbocation
\item déterminer excès énantiomérique par la loi de Biot
\item règles CIP
\end{itemize}

\paragraph*{Difficultés}

\begin{itemize}
\item vocabulaire,
\item représentation dans l'espace,
\item manipuler la loi de Biot
\end{itemize}


\section*{Introduction}
On a vu les différents types de réactions à l'aide de mécanismes réactionnels, on va s'intéresser ici à la géométrie des produits obtenus. Ceci est particulièrement important car scandale sanitaire \url{https://fr.wikipedia.org/wiki/Thalidomide#M\%C3\%A9canismes_de_l'effet_t\%C3\%A9ratog\%C3\%A8ne}.\\

\textcolor{green}{Expérience :} \url{https://spcl.ac-montpellier.fr/moodle/pluginfile.php/18438/mod_resource/content/3/Chapitre\%209\%20-\%20Aspets\%20microscopiques\%20-\%20Activite\%203.pdf}

\section{Chiralité: propriétés chimiques}

\subsection{Stéréoisomères de configuration}
Contenu: définition de molécules énantiomères et diastéréoisomères et exemples, comparaison des propriétés (biochimiques, chimiques et physiques)\\

\textcolor{blue}{Manipulation :} mesure température de fusion de deux diastéréoisomères: l’acide maléique et fumarique. L’acide maléique (Z) a une température de fusion inférieure à l’acide fumarique. Ceci s’explique par la présence d’une liaison hydrogène intramoléculaire diminuant ainsi le nombre de liaisons hydrogène intermoléculaires ce qui a pour effet de le rendre moins stable.

\subsection{Mélange racémique}
Contenu : définition, application en chimie organique (exemple de la substitution nucléophile d’ordre 1 qui conduit à un mélange racémique, attaque du carbocation plan par les deux côtés de façon équiprobable).


\section{Chiralité: propriétés physiques}

\subsection{Loi de Biot}
Contenu : activité optique d’une molécule chirale, énoncé de la loi de Biot, et additivité, lévogyre, dextrogyre.\\

\textcolor{blue}{Manipulation :} mesure du pouvoir rotatoire du limonène (R)ou (S)

\subsection{Excès énantiomérique}
Contenu : définition et calcul à partir d’un pouvoir rotatoire.\\

\textcolor{blue}{Manipulation bonus :} composition d’un mélange d’énantiomères de limonène (R) et (S) grâce à la mesure du pouvoir rotatoire total


\section{}

\subsection{}






\section{Conclusion} 


\end{reportBlock}





