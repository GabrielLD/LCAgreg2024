\begin{headerBlock}
\chapter{Structure des protéines et des lipides}
\label{LC_ControleQualite}
 \end{headerBlock}

%%%%%%%%%%%%%%%%%%%%%%%%%%%%%%%%%%%%%%%%%%%%%%%%%%%%
%%%% Références


%%%%%%%%%%%%%%%%%%%%%%%%%%%%%%%%%%%%%%%%%%%%%%%%%%%%
%%%% Plan
\begin{reportBlock}{Bibliographie}

\begin{center}
\begin{tabularx}{\textwidth}{| X | X | c | c |}\hline
Titre & Auteur(s) & Editeur (année) & ISBN \\ 
\hline
 \url{https://culturesciences.chimie.ens.fr/thematiques/chimie-organique/methodes-et-outils/les-acides-amines-et-la-synthese-peptidique} & Sur les acides aminés et liaison peptidique & & \\ 
 \hline
\end{tabularx}
\end{center}

\end{reportBlock}

\begin{reportBlock}{Plan détaillé}

\underline{Niveau} : Terminale ST2S \\

\section*{Introduction pédagogique}


\paragraph*{Prérequis}
\begin{itemize}
\item
\end{itemize}
\paragraph*{Contexte :}


\paragraph*{Notions importantes}

\begin{itemize}
\item 
\end{itemize}

\paragraph*{Objectifs}

\begin{itemize}
\item 
\end{itemize}

\paragraph*{Difficultés}

\begin{itemize}
\item 
\end{itemize}


\section*{Introduction }
Etiquette alimentaire montrant énergie, lipides, protéines en montrant lipides et protéines.
\textcolor{blue}{Manip qualitative : test lipides (gras) et protéines dans le lait caillé (lait + acide éthanoïque)}

\section{Structure des protéines}
Lancement de l'expérience principale sous hotte.

\subsection{Les acides $\alpha$-aminés}
Exemple : alamine : fonction acide carboxylique, fonction amine, carbone $\alpha$.\\

Définition acide aminé : molécule compos d'un groupement acide et d'un groupement amine. Il est dit $\alpha$-aminé si les deux groupes caratéristiques sont sur le carbone $\alpha$.\\

\textcolor{blue}{Manip qualitative : utiliser représentation moléculaire (Avogadro ou modèle réel) pour identifier le carbone asymétrique.}\\

Définition carbone asymétrique : tétraédrique et que le 4 groupes sur le carbone sont différents.\\
Les acides aminés sont des molécules chirales (image par un miroir plan non superposable).\\

Définition énantiomère : deux molécules chirales sont des énantiomères.\\

Slide représentation de Cram des acides aminé + représentation de Cram de l'alamine.

\subsection{La liaison peptidique}
Reaction de condensation : addition, élimination. Slide.

\subsection{Protéines}
Liaisons : 
\begin{itemize}
    \item hydrogène
    \item ionique
    \item pont dissulfure
    \item effet hydrophobe
\end{itemize}
Fonctions : 
\begin{itemize}
    \item structurale (formation des tissus humains, kératine),
    \item hormonal (insuline),
    \item transport (hémoglobine),
    \item enzymatique (lactase),
    \item immunitaire (anticorps)
\end{itemize}

\section{Structure des lipides}
\subsection{Acides gras}
Exemple : acide bytirique. Groupe carboxyle, chaîne carbonée.\\

Définition : un acide gras est un acide carboxylique associé à une chaîne carbonée.\\

Slide : \url{nutrixeal-info.fr} acide gras saturé, monoinsaturé, polyinsaturé.

\subsection{Triglycérides}
Exemple de triglycérides. Identifcation de groupement ester.\\

Pour fabriquer un triglycéride, il faut faire une estérification à partir d'acide gras, pour fabriquer un acide gras à partir d'un triglycéride, il faut faire une hydrolyse.\\

Equation d'une saponification : triglycéride + soude = glycérol + 3$C_{18}H_{23}O_2^-$.\\

\textcolor{blue}{Manip imposée : filtration sur Buchner du savon obtenu pendant la leçon. Précipitation du savon en le mettant dans de l'eau saturée en sel + de la glace. Puis filtration sur Büchner.}

\section*{Conclusion}
Finir sur les oligo-éléments.

\end{reportBlock}

\begin{reportBlock}{Remarques intéressantes}

   \begin{itemize}
       \item 
  \end{itemize}
    
\end{reportBlock}