\begin{headerBlock}
\chapter{Contrôle qualité de l'eau et de l'air}
\label{LC_ControleQualite}
 \end{headerBlock}

%%%%%%%%%%%%%%%%%%%%%%%%%%%%%%%%%%%%%%%%%%%%%%%%%%%%
%%%% Références


%%%%%%%%%%%%%%%%%%%%%%%%%%%%%%%%%%%%%%%%%%%%%%%%%%%%
%%%% Plan
\begin{reportBlock}{Bibliographie}

\begin{center}
\begin{tabularx}{\textwidth}{| X | X | c | c |}\hline
Titre & Auteur(s) & Editeur (année) & ISBN \\ \hline
 &  &  &  \\ 
 \hline
\end{tabularx}
\end{center}

\end{reportBlock}

\begin{reportBlock}{Plan détaillé}

\underline{Niveau} : Terminale ST2S \\

\section*{Introduction pédagogique}


\paragraph*{Prérequis}
\begin{itemize}
\item
\end{itemize}
\paragraph*{Contexte :}


\paragraph*{Notions importantes}

\begin{itemize}
\item 
\end{itemize}

\paragraph*{Objectifs}

\begin{itemize}
\item 
\end{itemize}

\paragraph*{Difficultés}

\begin{itemize}
\item 
\end{itemize}


\section*{Introduction }
Problématiques environnementales : problème sur la qualité de l'air (pollution dans les grandes villes par exemple) et de l'eau (algues verrtes en Bretagne par exemple). On doit avoir des moyens de contrôle de la qualité.

\section{Contrôle qualité de l'eau}
\subsection{Intérêt du controle de qualité}

Slide critère qualité de l'eau : microbiologique, organoleptique, limite de potabilité, limite de qualité.

\subsection{Solubilité des ions et de la conductivité}
Def dissolution et conductivité + \textcolor{blue}{manip qualitative pour la conductivité avec LED}.\\

Conductivité proportionnelle à la conentration des ions : \textcolor{blue}{Manip qualitative étalonnage solution ions chlorure.}\\

\subsection{Dosage conductimétrique pour déterminer la teneur en sulfate}
\textcolor{blue}{Manip quantitative : mesure de la quantité de $SO_4^{2-}$ dans la Contrex}. Réaction de titrage :
\begin{equation}
    SO_4^{2-}(aq) + Ba^{2+}(aq) = BaSO_4
\end{equation}
Avec $C_0(Ba^{2+})=5\times10^{-2}$~mol.L$^{-1}$. Tableau d'avancement avant et après équivalence.

\section{Contrôle de la qualité de l'air}

\subsection{Composition de l'air}
Slide composition de l'air.\\

Déf fraction molaire.

\subsection{Combustion d'une bougie}
\textcolor{blue}{Manip qualitative : allumer une bougie, la mettre sur un cristallisoir rempli d'eau, mettre une éprouvette dessus pour la piéger dans l'eau, voir que le niveau d'eau augmente au fur et à mesure que le diogène est consommé et que le dioxyde de carbone se dissout dans l'eau. On a une mesure de la quantité de CO2 ??}

\end{reportBlock}

\begin{reportBlock}{Remarques intéressantes}

   \begin{itemize}
       \item Si on boit de l'eau déminéralisée, pression osmotique qui va être plus forte de la cellule vers l'eau car la membrane cytoplasmique est poreuse : explosion de la cellule
       \item Si on boit de l'eau hyper salée, pression osmotique de l'eau vers la cellule : compression de la cellule
       \item la dureté de l'eau s’exprime en ppm (ou mg/L) de $CaCO_3$ ou en degrés français (symbole $^O$f ou  $^O$fH) en France et en Suisse. Un degré francais correspond à $10 ppm$ de calcaire représentant $10-4 mol L^{-1}$ de calcium, soit 4 mg/L de Ca2+, ou encore 2,4 mg de magnésium par litre d’eau (cf Wikipédia)
  \end{itemize}
    
\end{reportBlock}