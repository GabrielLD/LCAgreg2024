\begin{headerBlock}
\chapter{Oxydoréduction}
\label{LC_Oxydoreduction_STI2D}
 \end{headerBlock}

%%%%%%%%%%%%%%%%%%%%%%%%%%%%%%%%%%%%%%%%%%%%%%%%%%%%
%%%% Références


%%%%%%%%%%%%%%%%%%%%%%%%%%%%%%%%%%%%%%%%%%%%%%%%%%%%
%%%% Plan
\begin{reportBlock}{Bibliographie}

\begin{center}
\begin{tabularx}{\textwidth}{| X | X | c | c |}\hline
Titre & Auteur(s) & Editeur (année) & ISBN \\ \hline
 Chimie PCSI Chap 12 & J-B Baudin & Dunod (2019) &  \\ 
 \hline
 \url{https://webetab.ac-bordeaux.fr/Pedagogie/Physique/Physico/Electro/e02gener.htm} & Académie Bordeaux & & \\
 \hline
 Epreuve orale de Chimie p196-201 & F. Porteu-de-Buchère & Dunod (2017) & \\
 
\end{tabularx}
\end{center}

\end{reportBlock}

\begin{reportBlock}{Plan détaillé}

\underline{Niveau} : Terminale STL-SPCL \\

\section*{Introduction pédagogique}


\paragraph*{Prérequis}
\begin{itemize}
\item 
\end{itemize}
\paragraph*{Contexte :}


\paragraph*{Notions importantes}

\begin{itemize}
\item 
\end{itemize}

\paragraph*{Objectifs}

\begin{itemize}
\item 
\end{itemize}

\paragraph*{Difficultés}

\begin{itemize}
\item 
\end{itemize}

\section*{Introduction }

\section{Les biomolécules pour l'organisme}

\section{Conclusion} 

\end{reportBlock}