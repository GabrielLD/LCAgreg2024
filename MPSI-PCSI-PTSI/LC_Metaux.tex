\begin{headerBlock}
\chapter{Métaux : structure et propriétés}
\label{LC_Metaux}
 \end{headerBlock}

%%%%%%%%%%%%%%%%%%%%%%%%%%%%%%%%%%%%%%%%%%%%%%%%%%%%
%%%% Références


%%%%%%%%%%%%%%%%%%%%%%%%%%%%%%%%%%%%%%%%%%%%%%%%%%%%
%%%% Plan
\begin{reportBlock}{Bibliographie}

\begin{center}
\begin{tabular}{|c|c|c|c|}\hline
Titre & Auteur(s) & Editeur (année) & ISBN \\ \hline
PC Tout-en-un PSI/PSI* ~ & B. Fosset, J.-B. Baudin ~ & Dunod (2022) ~ & ~ \\
\hline
 &  &   ~ & ~ \\
\hline
\end{tabular}
\end{center}

\end{reportBlock}

\begin{reportBlock}{Plan détaillé}

\underline{Niveau} : MPSI \\

\underline{Pré-requis} :
\begin{itemize}
\item 
\end{itemize}


\section*{Introduction pédagogique}



\paragraph*{Prérequis}
\begin{itemize}
\item 
\end{itemize}

\paragraph*{Notions importantes}

\begin{itemize}
\item
\end{itemize}

\paragraph*{Objectifs}

\begin{itemize}
\item
\end{itemize}

\paragraph*{Difficultés}

\begin{itemize}
\item 
\end{itemize}

\section*{Introduction}
Métaux omniprésent dans le tableau périodique. Ils présentent des propriétés très sympathiques pour différentes applications (transport du courant (cuivre), étanchéification en cryogénie (indium), etc...). On va voir dans cette leçon les propriétés de structures et physico-chimiques des métaux.

\section{Cristaux métalliques}
\subsection{Propriétés physico-chimiques}
Cf Dunod p654.\\
\textcolor{blue}{Manip qualitative : utiliser un fil de cuivre pour montrer ses propriétés.}
\begin{itemize}
    \item Optique : \textbf{opaque au rayonnement électromagnétique}, réfléchit la lumière,
    \item Mécanique :  \textbf{ductile et maléable}, déformation sans rupture (cf indium dans l'étanchéité des cryostats),  \textbf{tenace} : résiste aux déformations et compressions (plaque de cuivre),
    \item Conduction : \textbf{conductivité électrique et thermique} très élevées
    \item Chimique : \textbf{bons réducteurs} (E$_{ionisation}$ faible) : forment des cations
\end{itemize}

\subsection{Modèle microscopique}
Modèle de Drude-Lorentz : les électrons forment un gaz parfait : les liaisons métalliques sont donc délocalisées = pas d'orientation particulière mais qui assurent la neutralité globale du cristal. Les cations sont fixes et forment un empilement compact de sphères dures. On parle du \textbf{modèle du cristal parfait} (voir Dunod PCSI p648) : ensemble de particule empilées tripériodiquement à l'infini et sans défaut dans l'espace.\\
\textcolor{blue}{Logiciel VESTA : struture cubique à faces centrées du cuivre groupe d'espace n°225}.

\section{Structures cristallines}
\textcolor{blue}{Logiciel VESTA : struture cubique à faces centrées du cuivre groupe d'espace n°225, parler de la maille cubique du cuivre}.
\subsection{Grandeurs caractéristiques d'une maille}
\begin{itemize}
    \item Coordinence : nombre de premiers voisins d'un atome du cristal, noté N
    \item Population : nombre d'atomes présents dans une maille cristalline
    \item Compacité : $C=\frac{Volume-occupe-par-les-atomes}{Volume-de-la-maille}$
\end{itemize}
\textcolor{red}{Transition : on va appliquer ces concepts en prenant l'exemple de la maille cubique à faces centrées du cuivre.}
\subsection{Structure cubique à faces centrées (cfc)}
\begin{itemize}
    \item Cfc : atomes aux n\oe uds de la maille et aux centres des faces. Principe de construction avec lempilement des plans de type ABC Dunod p667 \textcolor{blue}{Montrer sur le logiciel VESTA},
    \item $N=12$ (prendre un atome, compter le nombre de ses voisins), $Z=4$ faire la démo au tableau,
    \item Volume de la maille : $V_{maille}=a^3$,
    \item conditions de tangence : $4r = a\sqrt{2}$ \textcolor{blue}{Montrer sur le logiciel VESTA}
    \item Volume d'un atome : $V_{atome}=\frac{4\pi r^3}{3}=\frac{4\pi (a\sqrt{2})^3}{3\times4^3}$,
    \item Compacité : $C=\frac{N\times V_{1atome}}{V_{maille}}=0.74$ méthode à retenir,
    \item Masse volumique : $\rho_{Cu}=\frac{Zm_{Cu}}{a^3}=\frac{4M(Cu)}{N_Aa^3}$, avec $a_{Cu}^{tab}=362$~pm.
\end{itemize}
\textcolor{blue}{Manip quantitative : mesure de la masse volumique du cuivre.} Mettre des copeaux de cuivre dans une fiole jaugée, bien peser et faire la tare sur la balance. Ajouter l'eau, agiter car il y a des bulles collées sur les copeaux. Remplir la fiole précisément. Peser et en déduire le volume d'eau ajouter. Déterminer le volume de cuivre. La masse volumique est $\rho_{Cu}=\frac{m_{Cu}}{V_{Cu}}$, à comparer avec 8.960~g.cm$^{-3}$.\\

\subsection{Sites interstitiels}
Présenter les sites octaédriques et les sites tétraédriques. Parler de rayon d'habitabilité cf Dunod p671.

\section*{Conclusion}
Ouvrir sur les alliages : insertion et substitution. Très utilisé dans l'industrie car métaux pas si robustes que ça (ex : fer).\\

Ouvrir sur d'autres types de cristaux : ioniques (ex : NaCl, composés de ions de signes contraires, interactions coulombienne + principe d'excusion de Pauli), covalents (ex: graphite, diamant, liaisons covalentes entre les atomes (partage d'électrons pour stabiliser la configuration gaz noble)), moléculaires (ex : CO, H2, N2, glace, assemblés de molécules avec des liaisons de types VdW ou liaisons hydrogènes.


\end{reportBlock}

